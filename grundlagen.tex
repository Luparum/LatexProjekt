%!TEX root = thesis.tex

\chapter{Allgemeine Eigenschaften - Vergleich mit elektromagnetischen Wellen}
\label{chapter-basics}
\section{Erzeugung und Ausbreitungsgeschwindigkeit}
Nach der allgemeinen Relativitätstheorie wirken Änderungen des Gravitationsfeldes nicht instantan im ganzen Raum, wie es in der newtonschen Himmelsmechanik angenommen wird, sondern breiten sich mit Lichtgeschwindigkeit aus (siehe auch Aberration der Gravitation). Demnach werden von jedem System beschleunigter Massen (z. B. einem Doppelsternsystem oder einem um die Sonne kreisenden Planeten) Gravitationswellen erzeugt, ähnlich wie beschleunigte elektrische Ladungen elektromagnetische Wellen abstrahlen. Aufgrund des Birkhoff-Theorems sendet eine sphärisch symmetrisch oszillierende Massenverteilung keine Gravitationswellen aus (ebenfalls analog zur Elektrodynamik).

\section{Abwesenheit von Dipolwellen}
Die Masse ist die Ladung der Gravitation. Anders als bei der elektrischen Ladung gibt es keine negative Masse. Damit existieren keine Dipole von Massen. Ohne Dipole und ohne durch externe Kräfte hervorgerufene Bewegungen kann es jedoch keine Dipolstrahlung geben.


\section{Strahlung und Eichbosonen}
Gravitationswellen lassen sich mathematisch beschreiben als Fluktuationen des metrischen Tensors, eines Tensors 2. Stufe. Die Multipolentwicklung des Gravitationsfelds beispielsweise zweier einander umkreisender Sterne enthält als niedrigste Ordnung die Quadrupolstrahlung.
In einer quantenfeldtheoretischen Perspektive ergibt sich das der klassischen Gravitationswelle zugeordnete, die Gravitation vermittelnde Eichboson, das (hypothetische) Graviton, als Spin-2-Teilchen analog dem Spin-1-Photon in der Quantenelektrodynamik. Eine widerspruchsfreie quantenfeldtheoretische Formulierung der Gravitation auf allen Skalen ist jedoch noch nicht erreicht.

\section{Wellenart}
Gravitationswellen sind analog zu elektromagnetischen Wellen Transversalwellen. Aus Sicht eines lokalen Beobachters scheinen sie die Raumzeit quer zu ihrer Ausbreitungsrichtung zu stauchen und zu strecken. Sie haben ebenfalls zwei Polarisationszustände. Es gibt auch bei ihnen Dispersion.

\section{Mathematische Beschreibung}
Anders als für elektromagnetische Wellen – die sich aus den linearen Maxwell-Gleichungen ergeben – lässt sich eine Wellengleichung für Gravitationswellen nicht mehr exakt herleiten. Aus diesem Grunde ist auch das Superpositionsprinzip nicht anwendbar. Stattdessen gelten für Gravitationswellen die Einsteinschen Feldgleichungen. Für diese können in vielen Fällen nur Approximationslösungen durch lineare Differentialgleichungen ermittelt werden, z. B. die Wellengleichung als Näherung für kleine Amplituden. Dass die Annahme kleiner Amplituden am Entstehungsort der Welle in der Regel unzulässig ist, macht es sehr schwierig, die Abstrahlung von Gravitationswellen zu berechnen, was für Vorhersagen über die Messbarkeit der Wellen und die Gestalt der Signale jedoch erforderlich wäre. \\ \\
Aus der Nichtlinearität der Gravitationswellen folgt die Möglichkeit ihrer Darstellung als solitäre Wellenpakete.


