%!TEX root = thesis.tex

\chapter{Grundlagen}
\label{chapter-basics}

Dieses Kapitel beschreibt alle für die Arbeit notwendigen Grundlagen.

\section{Zielgruppe}

Die Zielgruppe einer Abschlussarbeit sind natürlich in erster Linie die Gutachter, die am Ende die Arbeit lesen und bewerten. Als Richtlinie, welches Wissen beim Leser einer solchen Arbeit vorausgesetzt werden kann, sollte man sich allerdings einen Kommilitonen des gleichen Studiengangs vorstellen, der in einem anderen Fachbereich seine Abschlussarbeit schreibt. Für diesen sollten wenigstens alle wesentlichen Definitionen enthalten sein.

\section{Umfang}

Eine Abschlussarbeit ist allerdings kein Lehrbuch. Entsprechend ist vor allem die Korrektheit wichtig. Zu umfangreiche Beispiele für in der Literatur bereits zur Genüge untersuchte Grundlagen sollten vermieden werden, um den Leser nicht zu langweilen.

\section{Quellen}

Ein wesentliches Charakteristikum von Grundlagen ist, dass diese nicht vom Autor der Arbeit erfunden wurden. Entsprechend ist gerade in den Grundlagen darauf zu achten, ausreichend Quellen anzugeben. Eine gute Regel ist dabei, immer den Erfinder bzw. das erste Auftauchen eines Konzeptes in der Literatur und mindestens eine gut verständliche neuere Quelle anzugeben. In den meisten Fällen hat sich seit der Einführung einer neuen Idee einiges getan, so dass neuere Quellen meistens einen besseren Einstieg in das Thema bieten. Es gehört sich aber, den Erfinder immer mit zu zitieren, da dieser die initiale Idee hatte.

