%!TEX root = thesis.tex

\chapter{Experimenteller Nachweis}
\label{chapter-fazit}
Die Effekte von Gravitationswellen sind derart klein, dass es auf absehbare Zeit nicht möglich sein wird, künstlich erzeugte Gravitationswellen nachzuweisen, sodass sie allenfalls nach astronomischen Ereignissen nachgewiesen werden können.
\section{Direkter Nachweis}
Heute werden Michelson-Interferometer verwendet, die hindurchwandernde Wellen in Echtzeit beobachten sollen, indem die lokalen Änderungen der Raumzeit-Eigenschaften die empfindliche Interferenz zweier Laserstrahlen verändern. Aktuelle Experimente dieser Art wie GEO600 (Deutschland/Großbritannien), VIRGO (Italien), TAMA 300 (Japan) und LIGO (USA) benutzen Lichtstrahlen, die in langen Tunneln hin- und herlaufen. Ein Unterschied in der Länge der Laufstrecke, wie er durch eine durchlaufende Gravitationswelle verursacht würde, könnte durch Interferenz mit einem Kontrolllichtstrahl nachgewiesen werden. Um auf diese Art eine Gravitationswelle direkt zu detektieren, müssen minimale Längenänderungen in Bezug auf die Gesamtlänge der Messapparatur – etwa 1/10.000 des Durchmessers eines Protons – festgestellt werden. Genauere Messungen auf größere Distanzen sollten zwischen Satelliten erfolgen. Das hierzu geplante Experiment LISA wurde 2011 von der NASA aus Kostengründen aufgegeben, wird aber vielleicht in kleinerem Maßstab von der ESA umgesetzt. Im Juli 2014 stellte die Universität von Tokio ihr „KAGRA“ (Kamioka Gravitational Wave Detector) genanntes Projekt in Hiba vor, das frühestens 2017 erste Ergebnisse liefern soll. Der Versuchsaufbau ähnelt dabei den in den USA und Europa zuvor verwendeten, soll aber um den Faktor 1000 genauer sein. \\\\
Am 11. Februar 2016 gaben Wissenschaftler den ersten direkten Nachweis von Gravitationswellen aus dem laufenden LIGO-Experiment bekannt. Das Ereignis wurde am 14. September 2015 nahezu zeitgleich mit 7 ms Differenz in den beiden LIGO-Observatorien in den USA beobachtet.
\begin{figure}
	\centering
	\pgfimage[width=.5\textwidth]{Messung}
	\caption[Interferometer]{Schematische Darstellung einer Inferometers.}
	\label{fig-flower}
\end{figure}
\newpage
Am 11. Februar 2016 gaben Wissenschaftler den ersten direkten Nachweis von Gravitationswellen aus dem laufenden LIGO-Experiment bekannt. Das Ereignis wurde am 14. September 2015 nahezu zeitgleich mit 7 ms Differenz in den beiden LIGO-Observatorien in den USA beobachtet.
\begin{figure}
	\centering\pgfimage[width=.5\textwidth]{Messergebnis}
	\caption[LIGO-Messung]{LIGO-Messergebnisse vom 14. September 2015}
\end{figure}

Ursachen für diese Gravitationswellen waren zwei Schwarze Löcher, welche umeinander rotierten bist diese schließlich kollidierten.

