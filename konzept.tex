%!TEX root = thesis.tex

\chapter{Quellen von Gravitationswellen}
\label{chapter-konzept}

Quellen intensiverer und damit nachweisbarer Gravitationswellen erwartet man bei Supernova-Explosionen sowie bei in geringem Abstand einander umkreisenden oder zusammenstoßenden Neutronensternen und/oder Schwarzen Löchern. \\ 

Nach der Art ihrer Quelle werden Gravitationswellen vier Kategorien zugeordnet:
\begin{enumerate}
	\item Kontinuierliche Gravitationswellen: Sie werden z. B. durch Neutronensterne verursacht. Bei konstanter Drehung verursachen sie eine in Frequenz und Amplitude konstante Gravitationswelle.
	\item Kompakte binäre spiralige Gravitationswellen: Kreisen zwei massereiche Objekte wie z. B. Weiße Zwerge, Neutronensterne oder Schwarze Löcher mit einem ihre Größe weit übersteigenden Abstand umeinander und bilden somit ein Paar mit einer bestimmten Umlaufbahn, so erzeugen sie charakteristische Gravitationswellen mit einer Dauer im Sekundenbereich. Durch diese Emissionen geht etwas Energie verloren, sodass über Millionen Jahre beobachtet die Umlaufbahn kleiner wird. Kleinere Umlaufbahnen bewirken durch den Pirouetteneffekt eine höhere Beschleunigung, was die Emission von Gravitationswellen steigert. Am Ende der spiraligen Phase erfolgt die Kollision, z. B. in einer Supernova.
	\item 
	Zufällige Gravitationswellen: Die vorgenannten Ereignisse treten aufgrund der geringen Dichte des Universums nur selten auf. Kleinere Quellen aus allen Richtungen können sich aber jederzeit bemerkbar machen.
	\item Ausbrechende Gravitationswellen: Signale von bisher noch nicht beschriebenen astronomischen Systemen werden hier eingeordnet.
	
\end{enumerate}
Jede Veränderung in der Verteilung von Masse und/oder Energie im Universum, bei der zumindest das Quadrupolmoment zeitlich variiert, führt zur Abstrahlung von Gravitationswellen. Im einfachsten Fall sind dies zwei einander umkreisende Massen. \\ 
So erzeugt der Umlauf der Erde um die Sonne Gravitationswellen, allerdings unmessbar schwache Wellen. Die abgestrahlte Leistung beträgt gerade einmal 300 W, weswegen auch die Beeinflussung der Erdbahn durch diesen Effekt nicht messbar ist. Um nur ein Millionstel der kinetischen Energie dieser Bewegung abzustrahlen, wären ungefähr $10^{18}$ (eine Trillion) Jahre nötig.
Auch der Urknall könnte Gravitationswellen angeregt haben, deren Frequenz aufgrund der kosmischen Expansion inzwischen jedoch sehr klein wäre. Der ursprünglich für das Jahr 2019 geplante Detektor eLISA wird diese möglicherweise nachweisen können. Nach dem Ausstieg der NASA ist die Zukunft des Projektes jedoch ungewiss. Das Folgeprojekt NGO (New Gravitational Wave Observatory) wurde von der europäischen Weltraumorganisation ESA zugunsten der Mission JUICE, deren Ziel die Erkundung der Jupitermonde ist, zurückgestellt.