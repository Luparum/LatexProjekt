%!TEX root = thesis.tex

\chapter{Einleitung}

Eine Gravitationswelle ist eine Welle in der Raumzeit, die durch eine beschleunigte Masse ausgelöst wird. Gemäß der allgemeinen Relativitätstheorie breitet sich jegliche Wirkung maximal mit Lichtgeschwindigkeit aus. Auch lokale Änderungen im Gravitationsfeld können sich also nur mit endlicher Geschwindigkeit ausbreiten. Daraus folgerte Albert Einstein 1916 die Existenz von Gravitationswellen. Diese verursachen während ihrer Ausbreitung durch den Raum vorübergehend Stauchungen und Streckungen von Abständen, also auch des Raumes selbst.
Da sich in der newtonschen Gravitationstheorie Veränderungen der Quellen des Gravitationsfeldes ohne Verzögerung im gesamten Raum auswirken, kennt diese keine Gravitationswellen.
Am 11. Februar 2016 berichteten Forscher der LIGO-Kollaboration über die erste erfolgreiche direkte Messung von Gravitationswellen im September 2015, die bei der Kollision zweier Schwarzer Löcher erzeugt wurden. Dies wird als Meilenstein in der Geschichte der Astronomie betrachtet


