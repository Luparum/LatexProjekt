\documentclass{scrbook}

%!TEX root = thesis.tex

% Set german to default language and load english as well
\usepackage[english,ngerman]{babel}

% Set UTF8 as input encoding
\usepackage[utf8]{inputenc}

% Set T1 as font encoding
\usepackage[T1]{fontenc}
% Load a slightly more modern font
\usepackage{lmodern}
% Use the symbol collection textcomp, which is needed by listings.
\usepackage{textcomp}
% Load a better font for monospace.
\usepackage{courier}

% Set some options regarding the document layout. See KOMA guide
\KOMAoptions{%
  paper=a4,
  fontsize=12pt,
  parskip=half,
  headings=normal,
  BCOR=1cm,
  headsepline,
  DIV=12}

% do not align bottom of pages
\raggedbottom

% set style of captions
\setcapindent{0pt} % do not indent second line of captions
\setkomafont{caption}{\small}
\setkomafont{captionlabel}{\bfseries}
\setcapwidth[c]{0.9\textwidth}

% set the style of the bibliography
\bibliographystyle{alphadin}

% load extended tabulars used in the list of abbreviation
\usepackage{tabularx}

% load the color package and enable colored tables
\usepackage[table]{xcolor}

% define new environment for zebra tables
\newcommand{\mainrowcolors}{\rowcolors{1}{maincolor!25}{maincolor!5}}
\newenvironment{zebratabular}{\mainrowcolors\begin{tabular}}{\end{tabular}}
\newcommand{\setrownumber}[1]{\global\rownum#1\relax}
\newcommand{\headerrow}{\rowcolor{maincolor!50}\setrownumber1}

% add main color to section headers
\addtokomafont{chapter}{\color{maincolor}}
\addtokomafont{section}{\color{maincolor}}
\addtokomafont{subsection}{\color{maincolor}}
\addtokomafont{subsubsection}{\color{maincolor}}
\addtokomafont{paragraph}{\color{maincolor}}

% do not print numbers next to each formula
\usepackage{mathtools}
\mathtoolsset{showonlyrefs}
% left align formulas
\makeatletter
\@fleqntrue\let\mathindent\@mathmargin \@mathmargin=\leftmargini
\makeatother

% Allow page breaks in align environments
\allowdisplaybreaks

% header and footer
\usepackage{scrpage2}
\pagestyle{scrheadings}
\setkomafont{pagenumber}{\normalfont\sffamily\color{maincolor}}
\setkomafont{pageheadfoot}{\normalfont\sffamily}
\setheadsepline{0.5pt}[\color{maincolor}]

% German guillemets quotes
\usepackage[german=guillemets]{csquotes}

% load TikZ to draw diagrams
\usepackage{tikz}

% load additional libraries for TikZ
\usetikzlibrary{%
  automata,%
  positioning,%
}

% set some default options for TikZ -- in this case for automata
\tikzset{
  every state/.style={
    draw=maincolor,
    thick,
    fill=maincolor!18,
    minimum size=0pt
  }
}

% load listings package to typeset sourcecode
\usepackage{listings}

% set some options for the listings package
\lstset{%
  upquote=true,%
  showstringspaces=false,%
  basicstyle=\ttfamily,%
  keywordstyle=\color{keywordcolor}\slshape,%
  commentstyle=\color{commentcolor}\itshape,%
  stringstyle=\color{stringcolor}}

% enable german umlauts in listings
\lstset{
  literate={ö}{{\"o}}1
           {Ö}{{\"O}}1
           {ä}{{\"a}}1
           {Ä}{{\"A}}1
           {ü}{{\"u}}1
           {Ü}{{\"U}}1
           {ß}{{\ss}}1
}

% define style for pseudo code
\lstdefinestyle{pseudo}{language={},%
  basicstyle=\normalfont,%
  morecomment=[l]{//},%
  morekeywords={for,to,while,do,if,then,else},%
  mathescape=true,%
  columns=fullflexible}

% load the AMS math library to typeset formulas
\usepackage{amsmath}
\usepackage{amsthm}
\usepackage{thmtools}
\usepackage{amssymb}

% load the paralist library to use compactitem and compactenum environment
\usepackage{paralist}

% load varioref and hyperref to create nicer references using vref
\usepackage[ngerman]{varioref}
\PassOptionsToPackage{hyphens}{url} % allow line break at hyphens in URLs
\usepackage{hyperref}

% setup hyperref
\hypersetup{breaklinks=true,
            pdfborder={0 0 0},
            ngerman,
            pdfhighlight={/N},
            pdfdisplaydoctitle=true}

% Fix bugs in some package, e.g. listings and hyperref
\usepackage{scrhack}

% define german names for referenced elements
% (vref automatically inserts these names in front of the references)
\labelformat{figure}{Abbildung\ #1}
\labelformat{table}{Tabelle\ #1}
\labelformat{appendix}{Anhang\ #1}
\labelformat{chapter}{Kapitel\ #1}
\labelformat{section}{Abschnitt\ #1}
\labelformat{subsection}{Unterabschnitt\ #1}
\labelformat{subsubsection}{Unterunterabschnitt\ #1}

% define theorem environments
\declaretheorem[numberwithin=chapter,style=plain]{Theorem}
\labelformat{Theorem}{Theorem\ #1}

\declaretheorem[sibling=Theorem,style=plain]{Lemma}
\labelformat{Lemma}{Lemma\ #1}

\declaretheorem[sibling=Theorem,style=plain]{Korollar}
\labelformat{Korollar}{Korollar\ #1}

\declaretheorem[sibling=Theorem,style=definition]{Definition}
\labelformat{Definition}{Definition\ #1}

\declaretheorem[sibling=Theorem,style=definition]{Beispiel}
\labelformat{Beispiel}{Beispiel\ #1}

\declaretheorem[sibling=Theorem,style=definition]{Bemerkung}
\labelformat{Bemerkung}{Bemerkung\ #1}

%!TEX root = thesis.tex

% Use this file to define some macros you need in your thesis. A macro is a short command that inserts some mathematical symbols or texts you do not want to retype each time you need some. I recommend to use as many macros as possible, because you are able to change them later. For example if you use the same macro each time you need to give the formal semantics of an expression you can easily change the appearance of these brackets by updating the macro later on.

% Set of natural numbers
\newcommand{\N}{\mathbb{N}}

% The default epsilon does not look very nice
\let\epsilon\varepsilon

% If you need to use mathematical expressins like an epsilon in the section titles of your thesis you will end up with warnings that these special symbols cannot be included in the PDF favorites. The following macro uses the mathematical symbol during the text of the thesis and the string "Epsilon" in the PDF favorites.
\newcommand{\pdfepsilon}{\texorpdfstring{$\epsilon$}{Epsilon}}


% Set title and author used in the PDF meta data
\hypersetup{
  pdftitle={Wie schreibe ich eine Masterarbeit?},
  pdfauthor={Erika Mustermann}
}

% Depending on which of the following two color schemes you import your thesis will be in color or grayscale. I recommend to generate a colored version as a PDF and a grayscale version for printing.

%!TEX root = thesis.tex

% define color of example university
\xdefinecolor{exampleuniversity}{rgb}{1, 0.5, 0}

\colorlet{maincolor}{exampleuniversity}

\colorlet{stringcolor}{green!60!black}
\colorlet{commentcolor}{black!50}
\colorlet{keywordcolor}{maincolor!80!black}

\newcommand{\imagesuffix}{-color}
%%!TEX root = thesis.tex

\colorlet{maincolor}{black}

\colorlet{stringcolor}{black}
\colorlet{commentcolor}{black!50}
\colorlet{keywordcolor}{black}

\newcommand{\imagesuffix}{-gray}

\newcommand{\duedate}{15. Juli 2016}

\begin{document}
  \frontmatter
  %!TEX root = thesis.tex

\begin{titlepage}
  \thispagestyle{empty}

  \vskip1cm

  \pgfimage[height=2.5cm]{uni-logo-example\imagesuffix}
  
  \vskip2.5cm
  
  \LARGE
  
  \textbf{\sffamily\color{maincolor}Wie schreibe ich eine Masterarbeit?}

  \textit{How to Write a Master Thesis?}

  \normalfont\normalsize

  \vskip2em
  
  \textbf{\sffamily\color{maincolor}Masterarbeit}

  im Rahmen des Studiengangs \\
  \textbf{\sffamily\color{maincolor}Informatik} \\
  der Universität zum Beispiel

  \vskip1em

  vorgelegt von \\
  \textbf{\sffamily\color{maincolor}Max Mustermann}

  \vskip1em
  
  ausgegeben und betreut von \\
  \textbf{\sffamily\color{maincolor}Prof. Dr. Erika Musterfrau}

  \vskip1em

  mit Unterstützung von\\
  Lieschen Müller

  \vskip1em

  Die Arbeit ist im Rahmen einer Tätigkeit bei der Firma Muster GmbH entstanden.


  \vfill

  Musterhausen, den \duedate
\end{titlepage}

  %!TEX root = thesis.tex

\cleardoublepage
\thispagestyle{plain}
\vspace*{\fill}

\section*{Erklärung}

Hiermit erkläre ich an Eides statt, dass ich die vorliegende
Arbeit ohne unzulässige Hilfe Dritter und ohne die Benutzung anderer
als der angegebenen Hilfsmittel selbständig verfasst habe;
die aus anderen Quellen direkt oder indirekt übernommenen Daten und Konzepte
sind unter Angabe des Literaturzitats gekennzeichnet.

\vskip2cm

\rule{5cm}{0.4pt}\\
(Max Mustermann)\\
Musterhausen, den \duedate

  %!TEX root = thesis.tex

\cleardoublepage
\thispagestyle{plain}

\pdfbookmark{Kurzfassung}{kurzfassung}
\paragraph{Kurzfassung} Diese Arbeit handelt um Gravitationswellen, wie diese entstehen, welche Quellen existieren und wie man diese nachweist.


\cleardoublepage
\thispagestyle{plain}

\foreignlanguage{english}{%
\pdfbookmark{Abstract}{abstract}
\paragraph{Abstract} This paper is about gravitational waves, how that arise, whats the source of that and how to show their existence.
}

  \cleardoublepage
  \phantomsection
  \pdfbookmark{Inhaltsverzeichnis}{tableofcontents}
  \markboth{Inhaltsverzeichnis}{}
  \tableofcontents

  \mainmatter
  %!TEX root = thesis.tex

\chapter{Einleitung}

Eine Gravitationswelle ist eine Welle in der Raumzeit, die durch eine beschleunigte Masse ausgelöst wird. Gemäß der allgemeinen Relativitätstheorie breitet sich jegliche Wirkung maximal mit Lichtgeschwindigkeit aus. Auch lokale Änderungen im Gravitationsfeld können sich also nur mit endlicher Geschwindigkeit ausbreiten. Daraus folgerte Albert Einstein 1916 die Existenz von Gravitationswellen. Diese verursachen während ihrer Ausbreitung durch den Raum vorübergehend Stauchungen und Streckungen von Abständen, also auch des Raumes selbst.
Da sich in der newtonschen Gravitationstheorie Veränderungen der Quellen des Gravitationsfeldes ohne Verzögerung im gesamten Raum auswirken, kennt diese keine Gravitationswellen.
Am 11. Februar 2016 berichteten Forscher der LIGO-Kollaboration über die erste erfolgreiche direkte Messung von Gravitationswellen im September 2015, die bei der Kollision zweier Schwarzer Löcher erzeugt wurden. Dies wird als Meilenstein in der Geschichte der Astronomie betrachtet



  %!TEX root = thesis.tex

\chapter{Allgemeine Eigenschaften - Vergleich mit elektromagnetischen Wellen}
\label{chapter-basics}
\section{Erzeugung und Ausbreitungsgeschwindigkeit}
Nach der allgemeinen Relativitätstheorie wirken Änderungen des Gravitationsfeldes nicht instantan im ganzen Raum, wie es in der newtonschen Himmelsmechanik angenommen wird, sondern breiten sich mit Lichtgeschwindigkeit aus (siehe auch Aberration der Gravitation). Demnach werden von jedem System beschleunigter Massen (z. B. einem Doppelsternsystem oder einem um die Sonne kreisenden Planeten) Gravitationswellen erzeugt, ähnlich wie beschleunigte elektrische Ladungen elektromagnetische Wellen abstrahlen. Aufgrund des Birkhoff-Theorems sendet eine sphärisch symmetrisch oszillierende Massenverteilung keine Gravitationswellen aus (ebenfalls analog zur Elektrodynamik).

\section{Abwesenheit von Dipolwellen}
Die Masse ist die Ladung der Gravitation. Anders als bei der elektrischen Ladung gibt es keine negative Masse. Damit existieren keine Dipole von Massen. Ohne Dipole und ohne durch externe Kräfte hervorgerufene Bewegungen kann es jedoch keine Dipolstrahlung geben.


\section{Strahlung und Eichbosonen}
Gravitationswellen lassen sich mathematisch beschreiben als Fluktuationen des metrischen Tensors, eines Tensors 2. Stufe. Die Multipolentwicklung des Gravitationsfelds beispielsweise zweier einander umkreisender Sterne enthält als niedrigste Ordnung die Quadrupolstrahlung.
In einer quantenfeldtheoretischen Perspektive ergibt sich das der klassischen Gravitationswelle zugeordnete, die Gravitation vermittelnde Eichboson, das (hypothetische) Graviton, als Spin-2-Teilchen analog dem Spin-1-Photon in der Quantenelektrodynamik. Eine widerspruchsfreie quantenfeldtheoretische Formulierung der Gravitation auf allen Skalen ist jedoch noch nicht erreicht.

\section{Wellenart}
Gravitationswellen sind analog zu elektromagnetischen Wellen Transversalwellen. Aus Sicht eines lokalen Beobachters scheinen sie die Raumzeit quer zu ihrer Ausbreitungsrichtung zu stauchen und zu strecken. Sie haben ebenfalls zwei Polarisationszustände. Es gibt auch bei ihnen Dispersion.

\section{Mathematische Beschreibung}
Anders als für elektromagnetische Wellen – die sich aus den linearen Maxwell-Gleichungen ergeben – lässt sich eine Wellengleichung für Gravitationswellen nicht mehr exakt herleiten. Aus diesem Grunde ist auch das Superpositionsprinzip nicht anwendbar. Stattdessen gelten für Gravitationswellen die Einsteinschen Feldgleichungen. Für diese können in vielen Fällen nur Approximationslösungen durch lineare Differentialgleichungen ermittelt werden, z. B. die Wellengleichung als Näherung für kleine Amplituden. Dass die Annahme kleiner Amplituden am Entstehungsort der Welle in der Regel unzulässig ist, macht es sehr schwierig, die Abstrahlung von Gravitationswellen zu berechnen, was für Vorhersagen über die Messbarkeit der Wellen und die Gestalt der Signale jedoch erforderlich wäre. \\ \\
Aus der Nichtlinearität der Gravitationswellen folgt die Möglichkeit ihrer Darstellung als solitäre Wellenpakete.



  %!TEX root = thesis.tex

\chapter{Quellen von Gravitationswellen}
\label{chapter-konzept}

Quellen intensiverer und damit nachweisbarer Gravitationswellen erwartet man bei Supernova-Explosionen sowie bei in geringem Abstand einander umkreisenden oder zusammenstoßenden Neutronensternen und/oder Schwarzen Löchern. \\ 

Nach der Art ihrer Quelle werden Gravitationswellen vier Kategorien zugeordnet:
\begin{enumerate}
	\item Kontinuierliche Gravitationswellen: Sie werden z. B. durch Neutronensterne verursacht. Bei konstanter Drehung verursachen sie eine in Frequenz und Amplitude konstante Gravitationswelle.
	\item Kompakte binäre spiralige Gravitationswellen: Kreisen zwei massereiche Objekte wie z. B. Weiße Zwerge, Neutronensterne oder Schwarze Löcher mit einem ihre Größe weit übersteigenden Abstand umeinander und bilden somit ein Paar mit einer bestimmten Umlaufbahn, so erzeugen sie charakteristische Gravitationswellen mit einer Dauer im Sekundenbereich. Durch diese Emissionen geht etwas Energie verloren, sodass über Millionen Jahre beobachtet die Umlaufbahn kleiner wird. Kleinere Umlaufbahnen bewirken durch den Pirouetteneffekt eine höhere Beschleunigung, was die Emission von Gravitationswellen steigert. Am Ende der spiraligen Phase erfolgt die Kollision, z. B. in einer Supernova.
	\item 
	Zufällige Gravitationswellen: Die vorgenannten Ereignisse treten aufgrund der geringen Dichte des Universums nur selten auf. Kleinere Quellen aus allen Richtungen können sich aber jederzeit bemerkbar machen.
	\item Ausbrechende Gravitationswellen: Signale von bisher noch nicht beschriebenen astronomischen Systemen werden hier eingeordnet.
	
\end{enumerate}
Jede Veränderung in der Verteilung von Masse und/oder Energie im Universum, bei der zumindest das Quadrupolmoment zeitlich variiert, führt zur Abstrahlung von Gravitationswellen. Im einfachsten Fall sind dies zwei einander umkreisende Massen. \\ 
So erzeugt der Umlauf der Erde um die Sonne Gravitationswellen, allerdings unmessbar schwache Wellen. Die abgestrahlte Leistung beträgt gerade einmal 300 W, weswegen auch die Beeinflussung der Erdbahn durch diesen Effekt nicht messbar ist. Um nur ein Millionstel der kinetischen Energie dieser Bewegung abzustrahlen, wären ungefähr $10^{18}$ (eine Trillion) Jahre nötig.
Auch der Urknall könnte Gravitationswellen angeregt haben, deren Frequenz aufgrund der kosmischen Expansion inzwischen jedoch sehr klein wäre. Der ursprünglich für das Jahr 2019 geplante Detektor eLISA wird diese möglicherweise nachweisen können. Nach dem Ausstieg der NASA ist die Zukunft des Projektes jedoch ungewiss. Das Folgeprojekt NGO (New Gravitational Wave Observatory) wurde von der europäischen Weltraumorganisation ESA zugunsten der Mission JUICE, deren Ziel die Erkundung der Jupitermonde ist, zurückgestellt.

\chapter{Experimenteller Nachweis}
\label{chapter-fazit}
Die Effekte von Gravitationswellen sind derart klein, dass es auf absehbare Zeit nicht möglich sein wird, künstlich erzeugte Gravitationswellen nachzuweisen, sodass sie allenfalls nach astronomischen Ereignissen nachgewiesen werden können.
\section{Direkter Nachweis}
Heute werden Michelson-Interferometer verwendet, die hindurchwandernde Wellen in Echtzeit beobachten sollen, indem die lokalen Änderungen der Raumzeit-Eigenschaften die empfindliche Interferenz zweier Laserstrahlen verändern. Aktuelle Experimente dieser Art wie GEO600 (Deutschland/Großbritannien), VIRGO (Italien), TAMA 300 (Japan) und LIGO (USA) benutzen Lichtstrahlen, die in langen Tunneln hin- und herlaufen. Ein Unterschied in der Länge der Laufstrecke, wie er durch eine durchlaufende Gravitationswelle verursacht würde, könnte durch Interferenz mit einem Kontrolllichtstrahl nachgewiesen werden. Um auf diese Art eine Gravitationswelle direkt zu detektieren, müssen minimale Längenänderungen in Bezug auf die Gesamtlänge der Messapparatur – etwa 1/10.000 des Durchmessers eines Protons – festgestellt werden. Genauere Messungen auf größere Distanzen sollten zwischen Satelliten erfolgen. Das hierzu geplante Experiment LISA wurde 2011 von der NASA aus Kostengründen aufgegeben, wird aber vielleicht in kleinerem Maßstab von der ESA umgesetzt. Im Juli 2014 stellte die Universität von Tokio ihr „KAGRA“ (Kamioka Gravitational Wave Detector) genanntes Projekt in Hiba vor, das frühestens 2017 erste Ergebnisse liefern soll. Der Versuchsaufbau ähnelt dabei den in den USA und Europa zuvor verwendeten, soll aber um den Faktor 1000 genauer sein. \\\\
Am 11. Februar 2016 gaben Wissenschaftler den ersten direkten Nachweis von Gravitationswellen aus dem laufenden LIGO-Experiment bekannt. Das Ereignis wurde am 14. September 2015 nahezu zeitgleich mit 7 ms Differenz in den beiden LIGO-Observatorien in den USA beobachtet.
\begin{figure}
	\centering
	\pgfimage[width=.5\textwidth]{Messung}
	\caption[Interferometer]{Schematische Darstellung einer Inferometers.}
	\label{fig-flower}
\end{figure}
\newpage
Am 11. Februar 2016 gaben Wissenschaftler den ersten direkten Nachweis von Gravitationswellen aus dem laufenden LIGO-Experiment bekannt. Das Ereignis wurde am 14. September 2015 nahezu zeitgleich mit 7 ms Differenz in den beiden LIGO-Observatorien in den USA beobachtet.
\begin{figure}
	\centering\pgfimage[width=.5\textwidth]{Messergebnis}
	\caption[LIGO-Messung]{LIGO-Messergebnisse vom 14. September 2015}
\end{figure}

Ursachen für diese Gravitationswellen waren zwei Schwarze Löcher, welche umeinander rotierten bist diese schließlich kollidierten.


  %!TEX root = thesis.tex

\chapter{Evaluierung}
\label{chapter-evaluation}

In der Evaluierung wird das Ergebnis dieser Arbeit bewertet. Eine praktische Evaluation eines neuen Algorithmus kann zum Beispiel durch eine Implementierung geschehen. Je nach Thema der Arbeit kann sich natürlich auch die gesamte Arbeit eher im praktischen Bereich mit einer Implementierung beschäftigen. In diesem Fall gilt es am Ende der Arbeit insbesondere die Implementierung selber zu evaluieren. Wesentliche Fragen dabei können sein:
\begin{compactitem}[--]
  \item Was funktioniert jetzt besser als vor meiner Arbeit?
  \item Wie kann das praktisch eingesetzt werden?
  \item Was sagen potenzielle Anwender zu meiner Lösung?
\end{compactitem}

\section{Implementierungen}

Wenn Implementierungen umfangreich beschrieben werden, ist darauf zu achten, den richtigen Mittelweg zwischen einer zu detaillierten und zu oberflächlichen Beschreibung zu finden. Eine Beschreibung aller Details der Implementierung ist in der Regel zu detailliert, da die primäre Zielgruppe einer Abschlussarbeit sich nicht im Detail in den geschriebenen Quelltext einarbeiten will. Die Beschreibung sollte aber durchaus alle wesentlichen Konzepte der Implementierung enthalten. Gerade bei einer Abschlussarbeit am Institut für Softwaretechnik und Programmiersprachen lohnt es sich, auf die eingesetzten Techniken und Programmiersprachen einzugehen. Ich würde in einer solchen Beschreibung auch einige unterstützende Diagramme erwarten.


  %!TEX root = thesis.tex

\chapter{Zusammenfassung und Ausblick}
\label{chapter-fazit}

Die Zusammenfassung greift die in der Einleitung angerissenen Bereiche wieder auf und erläutert, zu welchen Ergebnissen diese Arbeit kommt. Dabei wird insbesondere auf die neuen Erkenntnisse und den Nutzen der Arbeit eingegangen.

Im anschließenden Ausblick werden mögliche nächste Schritte aufgezählt, um die Forschung an diesem Thema weiter voranzubringen. Hier darf man sich nicht scheuen, klar zu benennen, was im Rahmen dieser Arbeit nicht bearbeitet werden konnte und wo noch weitere Arbeit notwendig ist.

  \appendix

  %!TEX root = thesis.tex

\chapter{Anhang}

Dieser Anhang enthält tiefergehende Informationen, die nicht zur eigentlichen Arbeit gehören.

\section{Abschnitt des Anhangs}

In den meisten Fällen wird kein Anhang benötigt, da sich selten Informationen ansammeln, die nicht zum eigentlichen Inhalt der Arbeit gehören. Vollständige Quelltextlisting haben in ausgedruckter Form keinen Wort und gehören daher weder in die Arbeit noch in den Anhang. Darüber hinaus gehören Abbildungen bzw. Diagramme, auf die im Text der Arbeit verwiesen wird, auf keinen Fall in den Anhang.

  \backmatter

  \cleardoublepage
  \phantomsection
  \pdfbookmark{Abbildungsverzeichnis}{listoffigures}
  \listoffigures

  \cleardoublepage
  \phantomsection
  \pdfbookmark{Tabellenverzeichnis}{listoftables}
  \listoftables

  \cleardoublepage
  \phantomsection
  \pdfbookmark{Definitions- und Theoremverzeichnis}{listoftheorems}
  \renewcommand{\listtheoremname}{Definitions- und Theoremverzeichnis}
  \listoftheorems[ignoreall,show={Lemma,Theorem,Korollar,Definition}]

  %!TEX root = thesis.tex

\cleardoublepage
\phantomsection
\pdfbookmark{Abkürzungsverzeichnis}{abbreviations}
\chapter*{Abkürzungsverzeichnis}
\label{section-abbrevs}

\begin{tabularx}{\textwidth}{lX}
  ABA & alternierender Büchi-Automat, engl. \emph{a}lternating \emph{B}üchi \emph{a}utomaton\\
  AFA & alternierender endlicher Automat, engl. \emph{a}lternating \emph{f}inite \emph{a}utomaton\\
  BA & Büchi-Automat, engl. \emph{B}üchi \emph{a}utomaton\\
  BNF & Normalform kontextfreier Grammatiken, engl. \emph{B}ackus--\emph{N}aur \emph{f}orm\\
  DFA & endlicher Automat, engl. \emph{d}eterministic \emph{f}inite \emph{a}utomaton
\end{tabularx}


  \cleardoublepage
  \phantomsection
  \pdfbookmark{Literaturverzeichnis}{bibliography}
  \bibliography{literature}
\end{document}
